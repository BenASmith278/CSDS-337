\documentclass[a4paper, 11pt]{article}
\usepackage{comment} % enables the use of multi-line comments (\ifx \fi) 
\usepackage{lipsum} %This package just generates Lorem Ipsum filler text. 
\usepackage{fullpage} % changes the margin
\usepackage[a4paper, total={7in, 10in}]{geometry}
\usepackage[fleqn]{amsmath}
\usepackage{amssymb,amsthm}  % assumes amsmath package installed
\newtheorem{theorem}{Theorem}
\newtheorem{corollary}{Corollary}
\usepackage{graphicx}
\usepackage{tikz}
\usetikzlibrary{arrows}
\usepackage{verbatim}
\usepackage[numbered]{mcode}
\usepackage{float}
\usepackage{tikz}
    \usetikzlibrary{shapes,arrows}
    \usetikzlibrary{arrows,calc,positioning}

    \tikzset{
        block/.style = {draw, rectangle,
            minimum height=1cm,
            minimum width=1.5cm},
        input/.style = {coordinate,node distance=1cm},
        output/.style = {coordinate,node distance=4cm},
        arrow/.style={draw, -latex,node distance=2cm},
        pinstyle/.style = {pin edge={latex-, black,node distance=2cm}},
        sum/.style = {draw, circle, node distance=1cm},
    }
\usepackage{xcolor}
\usepackage{mdframed}
\usepackage[shortlabels]{enumitem}
\usepackage{indentfirst}
\usepackage{hyperref}
    
\renewcommand{\thesubsection}{\thesection.\alph{subsection}}

\newenvironment{problem}[2][Problem]
    { \begin{mdframed}[backgroundcolor=gray!20] \textbf{#1 #2} \\}
    {  \end{mdframed}}

% Define solution environment
\newenvironment{solution}
    {\textit{Solution:}}
    {}

\renewcommand{\qed}{\quad\qedsymbol}
%%%%%%%%%%%%%%%%%%%%%%%%%%%%%%%%%%%%%%%%%%%%%%%%%%%%%%%%%%%%%%%%%%%%%%%%%%%%%%%%%%%%%%%%%%%%%%%%%%%%%%%%%%%%%%%%%%%%%%%%%%%%%%%%%%%%%%%%
\begin{document}
%Header-Make sure you update this information!!!!
\noindent
%%%%%%%%%%%%%%%%%%%%%%%%%%%%%%%%%%%%%%%%%%%%%%%%%%%%%%%%%%%%%%%%%%%%%%%%%%%%%%%%%%%%%%%%%%%%%%%%%%%%%%%%%%%%%%%%%%%%%%%%%%%%%%%%%%%%%%%%
\large\textbf{Benjamin Smith} \hfill \textbf{Problem Set - 6}   \\
Email: bxs566@case.edu \hfill ID: 3559750 \\
\normalsize Course: CSDS 337 - Compiler Design \hfill Term: Spring 2024\\
Instructor: Dr. Vipin Chaudhary \hfill Due Date: $24^{th}$ April, 2024 \\ \\
Number of hours delay for this Problem Set: \hfill 0.\\
Cumulative number of hours delay so far: \hfill 56. \\ \\
I discussed this homework with: \hfill No one. \\ \\
%\underline{\bf SUBMISSION GUIDELINES:} Submit a zip file that includes the %written answers and the flex file for Problem 4. \\

\noindent\rule{7in}{2.8pt}
%%%%%%%%%%%%%%%%%%%%%%%%%%%%%%%%%%%%%%%%%%%%%%%%%%%%%%%%%%%%%%%%%%%%%%%%%%%%%%%%%%%%%%%%%%%%%%%%%%%%%%%%%%%%%%%%%%%%%%%%%%%%%%%%%%%%%%%%
% Problem 1
%%%%%%%%%%%%%%%%%%%%%%%%%%%%%%%%%%%%%%%%%%%%%%%%%%%%%%%%%%%%%%%%%%%%%%%%%%%%%%%%%%%%%%%%%%%%%%%%%%%%%%%%%%%%%%%%%%%%%%%%%%%%%%%%%%%%%%%%
\begin{problem}{1 - 10 points}
Generate code for the following three-address statements assuming \verb!a! and \verb!b! are arrays whose elements are 4-byte values. The four-statement sequence
\begin{verbatim}
x = a[i]  
y = b[j]  
a[i] = y  
b[j] = x  
\end{verbatim}

\end{problem}
\begin{solution}
\end{solution}

\noindent\rule{7in}{2.8pt}

%%%%%%%%%%%%%%%%%%%%%%%%%%%%%%%%%%%%%%%%%%%%%%%%%%%%%%%%%%%%%%%%%%%%%%%%%
% Problem 2
%%%%%%%%%%%%%%%%%%%%%%%%%%%%%%%%%%%%%%%%%%%%%%%%%%%%%%%%%%%%%%%%%%%%%%%%%%%%%%%%%%%%%%%%%%%%%%%%%%%%%%%%%%%%%%%%%%%%%%%%%%%%%%%%%%%%%%%%

\begin{problem}{2 - 10 points}
Determine the cost of executing the following.
\begin{enumerate}
    \item \begin{verbatim}
        LD R0, c  
        LD R1, i  
        MUL R1, R1, 8  
        ST a(R1), R0  
    \end{verbatim}
    \item \begin{verbatim}
        LD R0, p  
        LD R1, 0(R0) 
        ST x, R1  
    \end{verbatim}
\end{enumerate}
\end{problem}
\begin{solution}
\end{solution}

\noindent\rule{7in}{2.8pt}

%%%%%%%%%%%%%%%%%%%%%%%%%%%%%%%%%%%%%%%%%%%%%%%%%%%%%%%%%%%%%%%%%%%%%%%%%
% Problem 3
%%%%%%%%%%%%%%%%%%%%%%%%%%%%%%%%%%%%%%%%%%%%%%%%%%%%%%%%%%%%%%%%%%%%%%%%%%%%%%%%%%%%%%%%%%%%%%%%%%%%%%%%%%%%%%%%%%%%%%%%%%%%%%%%%%%%%%%%

\begin{problem}{3 - 10 points}
Below is code to count the number of primes from 2 to  n , using the sieve method on a suitably large array a. That is, a [i] is TRUE at  the end only if there is no prime $\sqrt i$ or less that evenly divides i. We initialize  all a[i] to TRUE and then set a[j] to FALSE if we find a divisor of j.

\begin{verbatim}
for (i=2; i<=n; i++)    
    a[i] = TRUE;      
count = 0;     
s = sqrt(n);     
for (i=2; i<=s; i++)    
    if (a[i]) /* i has been found to be a prime */ {}  
        count++;    
        for (j=2*i; j<=n; j = j+i)    
            a[j] = FALSE; /* no multiple of i is a prime */  
    }
\end{verbatim}

\begin{enumerate}[a]
    \item Translate the program into three-address statements of the type we have  been using in this section. Assume integers require 4 bytes.
    \item Construct the flow graph for your code from (a)
    \item Identify the loops in your flow graph from (b).

\end{enumerate}
\end{problem}
\begin{solution}
\end{solution}

\noindent\rule{7in}{2.8pt}
%%%%%%%%%%%%%%%%%%%%%%%%%%%%%%%%%%%%%%%%%%%%%%%%%%%%%%%%%%%%%%%%%%%%%%%%%
%%%%%%%%%%%%%%%%%%%%%%%%%%%%%%%%%%%%%%%%%%%%%%%%%%%%%%%%%%%%%%%%%%%%%%%%%
% Problem 4
%%%%%%%%%%%%%%%%%%%%%%%%%%%%%%%%%%%%%%%%%%%%%%%%%%%%%%%%%%%%%%%%%%%%%%%%%%%%%%%%%%%%%%%%%%%%%%%%%%%%%%%%%%%%%%%%%%%%%%%%%%%%%%%%%%%%%%%%

\begin{problem}{4 - 10 points}
Suppose a basic block is formed from the C assignment statements
\begin{verbatim}
    x = a + b + c + d + e + f;  
    y = a + c + e;
\end{verbatim}

\begin{enumerate}[a]
    \item Give the three-address statements (only one addition per statement) for  this block.
    \item Use the associative and commutative laws to modify the block to use the  fewest possible number of instructions, assuming both x and y are live on  exit from the block.
\end{enumerate}

\end{problem}

\begin{solution}
\end{solution}


\noindent\rule{7in}{2.8pt}
%%%%%%%%%%%%%%%%%%%%%%%%%%%%%%%%%%%%%%%%%%%%%%%%%%%%%%%%%%%%%%%%%%%%%%%%%
%%%%%%%%%%%%%%%%%%%%%%%%%%%%%%%%%%%%%%%%%%%%%%%%%%%%%%%%%%%%%%%%%%%%%%%%%
% Problem 5
%%%%%%%%%%%%%%%%%%%%%%%%%%%%%%%%%%%%%%%%%%%%%%%%%%%%%%%%%%%%%%%%%%%%%%%%%%%%%%%%%%%%%%%%%%%%%%%%%%%%%%%%%%%%%%%%%%%%%%%%%%%%%%%%%%%%%%%%

\begin{problem}{5 - 20 points}
Consider the expression $(a-b) + e * (c+d)$.
\begin{enumerate}[a]
    \item Generate optimized code using three registers.
    \item Generate optimized code using two registers.
\end{enumerate}

\end{problem}

\begin{solution}
\end{solution}

\noindent\rule{7in}{2.8pt}

\end{document}
